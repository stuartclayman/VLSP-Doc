% This needs a few passes to layout correctly

{
  \renewcommand{\arraystretch}{1.6} 
  \renewcommand{\tabcolsep}{1.1ex}


\small

% longtable lines
\begin{longtable}{ | p{7.5cm} | p{6.5cm} | }

\caption{Global Controller Configuration Options} \\
  \hline \multicolumn{2}{| c |} {\textbf{Configuration Options}}\\
  \hline  {\textbf{Field}} & {\textbf{Description}} \\
\endhead




\hline
\footnotesize{\texttt{<SimOptions>} (required)} & The top level node for all
configuration options to the Global Controller. \\
\hline \hline

% GlobalController

\hline
\footnotesize{\texttt{<GlobalController>} (optional)} &
The node for specific options for  the Global Controller.  Only
one such section is allowed.
\\
\hline
\footnotesize{\texttt{<GlobalController> $\rightarrow$ <Simulation> (optional)}} &
If true, the Global Controller simulates routers rather than starting
them on local controllers.
\newline
Boolean, default is \texttt{false}. 
 \\
\hline
\footnotesize{\texttt{<GlobalController> $\rightarrow$ <Port> (optional)}} &
The port that the Global Controller should listen on. 
\newline
Integer value -- default is \texttt{8888}.
\\

\hline
\footnotesize{\texttt{<GlobalController> $\rightarrow$ <StartLocalControllers> (optional)}} &
Should the the Global Controller start any localcontrollers, if
necessary. 
\newline
Boolean, default is \texttt{true}.  If false GC assumes local controllers
started by hand.
 \\

\hline
\footnotesize{\texttt{<GlobalController> $\rightarrow$ <ConnectedNetwork> (optional)}} &
If true the GC will add a link to connect the network if it becomes
disconnected.
\newline
Boolean, default is \texttt{false}.
\\
\hline
\footnotesize{\texttt{<GlobalController> $\rightarrow$ <AllowIsolatedNodes> (optional)}} &
If false the GC will reconnect a node which has no links.
\newline
Boolean, default is \texttt{true}.
\\
\hline
\footnotesize{\texttt{<GlobalController> $\rightarrow$ <PlacementEngineClass> (optional)}} &
The name of the class of the placement engine in the Global Controller. 
\newline
String, default is \texttt{usr.globalcontroller. LeastUsedLoadBalancer}.
\\
\hline
\footnotesize{\texttt{<GlobalController> $\rightarrow$ <VisualizationClass> (optional)}} &
The name of the class for Visualization in the Global Controller. 
\newline
String, there is no default.
\\
\hline
\footnotesize{\texttt{<GlobalController> $\rightarrow$ <Monitoring> (optional)}} &
Should the Global Controller start the Lattice monitoring
framework on the routers. 
\newline
Boolean, default is \texttt{false}.
\\
\hline
\footnotesize{\texttt{<GlobalController> $\rightarrow$ <RemoteLoginUser> (optional) }} &
Provide a user name to create ssh tunnel when starting local
controllers (can be overridden per controller)
\newline
String, no default.
\\
\hline
\footnotesize{\texttt{<GlobalController> $\rightarrow$ <LowPort>} (optional)} &
Lowest port to be used for listening by local controllers (can
be overridden by individual controllers)
\newline
Integer, default is \texttt{10000}.
\\
\hline
\footnotesize{\texttt{<GlobalController> $\rightarrow$ <HighPort> (optional)}} &
Highest port to be used for listening by local controllers (can
be overridden by individual controllers)
\newline
Integer, default is \texttt{20000}.
\\
\hline 
\hline

% LocalControllers

\hline 
\footnotesize{\texttt{<LocalController> (optional)}} &
The node for specific options for  the Local Controller. 
One such section should be present for every physical
machine in the test bed.  (For simulation there should be none.)\\

\hline
\footnotesize{\texttt{<LocalController> $\rightarrow$ <Name> (compulsory)}} &
The name of the host that the Local Controller should run on. \newline
String (no default)
\\

\hline
\footnotesize{\texttt{<LocalController> $\rightarrow$ <Port> (compulsory)}} &
The port that the Local Controller should listen on. \newline
Integer (no default)
\\
\hline
\footnotesize{\texttt{<LocalController> $\rightarrow$ <LowPort> (optional)}} &
The lowest port number that the Local Controller should start a Router
listening on. \newline
Integer (default inherited from GlobalController)
\\

\hline
\footnotesize{\texttt{<LocalController> $\rightarrow$ <HighPort> (optional)}} &
The highest port number that the Local Controller should start a Router
listening on. \newline
Integer (default inherited from GlobalController) \\

\hline
\footnotesize{\texttt{<LocalController> $\rightarrow$ <MaxRouters>}} &
The maximum number of Routers that a Local Controller should start 
on the host. \newline
Integer (default 100) \\

\hline
\footnotesize{\texttt{<LocalController> $\rightarrow$ <RemoteLoginUser>}} &
The username used to login to this controller via ssh to start LocalController \newline
String (no default -- use current user as default) but can be inherited from GC \\
\hline
\footnotesize{\texttt{<LocalController> $\rightarrow$ <RemoteStartController>}} &
Command string used to start LocalController on machine \newline
String (no default but see below) \newline
if not set use \texttt{java -cp [Classpath from environment] usr.localcontroller.LocalController}\\
\hline 
\hline

% EventEngine

\hline
\footnotesize{\texttt{<EventEngine>}} &
The node for specific options for  the Event engine. \\

\hline 
\footnotesize{\texttt{<EventEngine> $\rightarrow$ <Name>}} &
The name of the probability distribution function. \\
\hline
\hline

% Routeroptions

\hline
\footnotesize{\texttt{<RouterOptions>}} &
The node for specific options for  the routers. \\
\hline \hline

% Output

\hline
\footnotesize{\texttt{<Output>}} &
The node for specific options for  generating output from a run. \\
\hline

\end{longtable}


\normalsize

}
