In this appendix there is a description of the datagrams that are sent
by the routers. First there is a datagram outline, then the datagram
elements are described in more detail.

\begin{table}[h!]
{
  \renewcommand{\arraystretch}{1.2} 
  \renewcommand{\tabcolsep}{1.2ex}

  \small

%\begin{tabular} { | l | l | l | l | l | l | l | l | l | l | l | l | l | l | l | }
\begin{tabular} { | l | l | l | l | l | l | l | l | l | l | l | }
\hline
{\small{USRD}}%
& \parbox[t]{0.7cm}{hdr\\len}%
& \parbox[t]{0.7cm}{total\\len}%
& flags%
& ttl%
& \parbox[t]{0.9cm}{proto\\col}%
& \parbox[t]{0.7cm}{src\\addr}%
& \parbox[t]{0.7cm}{dst\\addr}%
& \textit{pad} %
& \parbox[t]{0.7cm}{src\\port}%
& \parbox[t]{0.7cm}{dst\\port} \\% 
\hline
\end{tabular} \newline
$\hookrightarrow$ $\quad$
\begin{tabular} { | l | l | l | l |}
  \parbox[t]{0.9cm}{time\\stamp}%
& \parbox[t]{0.7cm}{flow\\id}%
& \parbox[t]{0.7cm}{pay\\load}%
& \parbox[t]{0.8cm}{check\\sum} \\
\hline
\end{tabular}

\normalsize

}

\end{table}

\noindent The following table \ref{datagram:elements} presents each of
these elements in more
detail, by showing the number of bytes each element takes, and giving
a description of the element.


{
  \renewcommand{\arraystretch}{1.6} 
  \renewcommand{\tabcolsep}{1.1ex}

  \small

\begin{longtable}{ | l | l | l | p{9.5cm} | }

\caption{Datagram Elements} \label{datagram:elements}\\

\hline  \multicolumn{4}{| c |} {\textbf{Datagram Elements}}\\

\hline  {\textbf{Element}} & {\textbf{Bytes}} & {\textbf{Position}} & {\textbf{Description}} \\

\endhead

\hline

USRD & 4 & 0 & The literal String ``USRD''. \\
\hline

Hdr Len & 1 & 4 & The size, in bytes, of the header.\newline
This currently set to 36. \\
\hline

Total Len & 2 & 5 & The size, in bytes, of the whole Datagram. \\
\hline

Flags & 1 & 7 & Some optional flags. \newline
This gives 8 bits to play with. \\
\hline

TTL & 1 & 8 & The \textsc{TTL} of a Datagram. \\
\hline

Protocol & 1 & 9 & The protocol specified for a particular Datagram.\newline
Currently there is a \textsc{CONTROL} protocol and a
\textsc{DATA} protocol.  \\
\hline

Src Addr & 4 & 10 & The source address of the Datagram. \\
\hline

Dst Addr & 4 & 14 & The destination address of the Datagram. \\
\hline

Pad & 2 & 18 & Some spare bytes, currently padded as 2 spaces {$\bigtriangledown\bigtriangledown$}. \\
\hline

Src Port & 4 & 20 & The source port of the Datagram. \\
\hline

Dst Port & 4 & 22 & The destination port of the Datagram. \\
\hline

Timestamp & 8 & 24 & The time the Datagram was created. \\
\hline

Flow ID & 4 & 32 & A flow ID. \\
\hline

Payload & $N$ &  36 & The payload itself.  \newline
It's size, $N$, is Total Len $-$ Hdr Len $-$ 4. \\
\hline

Checksum & 4 & $ 36 + N$ & The checksum. \\
\hline

\end{longtable}

\normalsize

}


\noindent The following table presents the methods that can be used to
manipulate a Datagram object.

\emph{TODO}
