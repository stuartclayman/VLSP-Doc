In this Appendix there is a description of the MCRP protocol, with
explaining each request and response.

% This needs a few passes to layout correctly

In this Appendix there is a description of the MCRP protocol, with
explaining each request and response.


The commands that can be sent to a Router through its Management
Console are shown here.
These commands affect the router in various ways, where some are quite
lightweight (such as getting the Router's name), and others are more
heavyweight (such as connecting the Router to another Router).

Other components connect to the router via the Management Console,
acting in essence as a client, to interact with the Router.
In most instances, a permanent connection will be made by the Local Controller
on the machine that runs the Router.   The Router also accepts
multiple connections from any number of elements. It is  usually other
Routers thay connect to Routers in order to create new virtual network connections.
Care needs
to be taken if multiple elements send control commands.

The following table shows each of these commands.

% This needs a few passes to layout correctly
{
  \renewcommand{\arraystretch}{1.2} 
  \renewcommand{\tabcolsep}{1.1ex}


% longtable lines
\begin{longtable}[l]{ | l | p{10cm} | }

 \caption{Router Commands} \\

 \hline \multicolumn{2}{| c |} {\textbf{Router Commands}}\\

 \hline  {\textbf{Command}} & {\textbf{Description}}\\

 \endhead
\hline


GetName & Gets the name of the Router. 
\\
\hline

SetName & Sets the name of the Router.
\\
\hline

GetRouterAddress & Gets the address of the Router.
\\
\hline

SetRouterAddress & Sets the address of the Router.
\\
\hline

GetConnectionPort & Gets the port that the Router listens on in
order to accept new Router to Router virtual network connections.
\\
\hline

ListConnections & Lists all of the virtual network connections that
the Router has.
\\
\hline

IncomingConnection & Tells the Router that a new virtual network
connection \emph{has} already come in from another specified Router.  \newline
This command comes from another Router, and is the second phase of
Router to Router connections. \\
\hline

CreateConnection & Tells the Router that a new virtual network
connection is coming in from another specified Router.  \newline
This command comes from another Router, and is the first phase of
Router to Router connections. \\
\hline

SetLinkWeight & Set the weight of a specified virtual network
connection / link between two Routers.
\hline

EndLink & End a specified virtual network connection / link between two Routers. \\
\hline

ListRoutingTable & Lists the whole routing table. \\
\hline

GetPortName & Gets the name of a specified port. \\
\hline

GetPortAddress & Get the address associated with a specified port.\\
\hline

SetPortAddress & Set the address associated with a specified port.\\
\hline

GetPortWeight & Get the weight associated with a specified port.\\
\hline

SetPortWeight & Set the weight associated with a specified port.\\
\hline

GetPortRemoteRouter & Gets the name of the remote router, at the other
end of a virtual network connection, on a specified port.
\\
\hline

GetPortRemoteAddress &  Gets the address of the remote router, at the other
end of a virtual network connection, on a specified port.
\\
\hline

AppStart & Start an application on the Router. \\
\hline

AppStop & Stop an application. \\
\hline

AppList & List all the applications running on the Router. \\
\hline

GetNetIFStats & Get the statistics for all the $network interfaces$ on
the Router. \\
\hline

GetSocketStats & Get the statistics for all the $sockets$ on
the Router.\\
\hline

MonitoringStart & Start the monitoring sub-system and all the
configured probes on the Router. \\
\hline

MonitoringStop & Stop the monitoring sub-system and all the
configured probes on the Router. \\
\hline

SetAP & Set the Aggregation Point config for a Router. \\
\hline

ReadOptionsFile & Read the configuration options for routers from a
file, e.g. routeroptions.xml. \\
\hline

ReadOptionsString & Read the configuration options for routers, passed
in a a big string. \\
\hline

ShutDown & Tells the Router to shutdown.  All applications will be
stopped, all network connections to other Routers will be closed, and
every component of the router will clean up. \\
\hline

RouterOK & Asks the router if it is OK.\\
\hline

% spurious other stuff
%% Ping & \\
%% \hline

%% Echo & \\
%% \hline

%% Run \\
%% \hline


\end{longtable}  

}



%% {
%%   \renewcommand{\arraystretch}{1.6} 
%%   \renewcommand{\tabcolsep}{1.1ex}


%% \small

%% % longtable lines
%% \begin{longtable}{ | p{4cm}  p{3cm} | l | l | }

%%  \caption{Generic MCRP Protocol Values} \\
%%  \hline \multicolumn{4}{| c |} {\textbf{Generic MCRP Protocol Values}}\\
%%  \hline  {\textbf{Request}} & & {\textbf{Success Code}} & {\textbf{Error Code}}\\
%%  \endhead

%% \hline


%% \smalltt{QUIT} & & 200 & 400 \\
%% \hline & \multicolumn{3}{| p{9cm} |}{
%% Quit talking to the Component.
%% It closes a connection to the Management Console of the Component.
%% } \\
%% \hline 
%% \multicolumn{4}{| p{9cm} |} {
%% \emph{Request:} \smalltt{QUIT} \newline
%% \emph{Response:} \smalltt{200 BYE}
%% } \\

%% \hline \hline

%% \smalltt{SHUT\_DOWN} & & 201 & 401 \\
%% \hline & \multicolumn{3}{| p{9cm} |} {
%% Shut down the Component.
%% } \\

%% \hline \hline

%% \smalltt{ON\_ROUTER} & & 202 & 401 \\
%% \hline & \multicolumn{3}{| p{9cm} |} {
%% Send an MCRP request for the specified Router.
%% \newline
%% \smalltt{ON\_ROUTER router\_id className args}
%% } \\
%% \hline \multicolumn{4}{| p{9cm} |} {
%% \emph{Request:} \smalltt{ON\_ROUTER 1 usr.applications.Ping 2}
%% } \\
%% \hline \hline

%% \hline

%% \end{longtable}

%% \normalsize
%% }

%% {
%%   \renewcommand{\arraystretch}{1.6} 
%%   \renewcommand{\tabcolsep}{1.1ex}


%% \small

%% \begin{longtable}{ | p{4cm}  p{3cm} | l | l | }

%%  \caption{Router Specific MCRP Protocol Values} \\
%%  \hline \multicolumn{4}{| c |} {\textbf{Router Specific MCRP Protocol Values}}\\
%%  \hline  {\textbf{Request}} & & {\textbf{Success Code}} & {\textbf{Error Code}}\\
%%  \endhead

%% \hline

%% \smalltt{GET\_NAME} & & 221 & 400 \\
%% \hline & \multicolumn{3}{| p{9cm} |} {
%% Get the name of a router.
%% } \\
%% \hline
%% \multicolumn{4}{| p{9cm} |} {
%% \emph{Request:} \smalltt{GET\_NAME} \newline
%% \emph{Response:} \smalltt{221 Router-17}
%% } \\

%% \hline \hline

%% \hline

%% \end{longtable}



%% \normalsize
%% }
