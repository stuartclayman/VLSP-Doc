
In this section the usage of the platform is described.  We show how
to start up the platform and how to configure different runs.

Remember that there are 3 main components of the platform:

\begin{itemize}
\item the Global Controller
\item Local Controllers
\item Routers
\end{itemize}

\noindent Within a run, there will be one Global Controller, which controls the
run, a Local Controller on each physical machine that participates,
and a set of Routers that run on the physical machines.

To start the whole VLSP system, we execute the \emph{Global Controller} and pass in an \textsc{xml} file which has the configuration options
for each particular run, specifying
 a Java class
called:  \lstinline[language=Java]$usr.globalcontroller.GlobalController$.
The options include information such as the
ports to listen on; the hosts that the \emph{Host Controller} is to execute
on and its associated ports; its monitoring elements; the placement engine to use; and other
virtual entity options.  When the \emph{Global
Controller} starts it attempts to setup all of the monitoring elements
and start all of the \emph{Host Controllers} on
the specified hosts, this  is a Java class
called:  \lstinline[language=Java]$usr.localcontroller.LocalController$.  Once
this has been achieved the VLSP system is ready for experiments.

The interface between all the VLSP components uses \textsc{http}.
All communicated information is transmitted using
\textsc{rest} and \textsc{json} descriptions. To maintain the
lightness of the system we use the \texttt{monoid Resty} jar library,
which is only 120K.  This compares to the commonly used, but fully
featured, Jersey library which is 1.5Mb.

The start up and shutdown of virtual routers is managed by the \emph{Global
Controller} but is performed by the \emph{Host Controller} which
resides on each host after receiving \textsc{rest} calls.
The \emph{Host Controller} is also
used to control
the connection of virtual routers with virtual links.
The \emph{Host Controller}  behaves in the same way a hypervisor does
in other virtualised environments, and it also passes on \emph{Global
Controller} commands to Routers.
A virtual entity will be started on the same physical machine as the
\emph{Host Controller}.




\subsection{Pre-Configuration}

As the platform is
written in Java, we need to set the JAVA\_HOME and the CLASSPATH
environment variables.

\code{\% export JAVA\_HOME=/usr/java/jdk1.8/}

\noindent To start the run of the platform, either set the current directory
to be where the platform is installed and the relevant environment variables
need to be set before everything is started:

\code{\% cd /install\_place}

\code{\% export CLASSPATH=.:./libs/*:}

\noindent or set the install\_place in the classpath:

\code{\% export CLASSPATH=/install\_place:/install\_place/libs/*:}

\subsection{Starting}

The Global Controller is started and passed an XML
configuration control file -- the path to the configuration file can
be relative or absolute. 

\code{\% java usr.globalcontroller.GlobalController control-config.xml}

\noindent or:

\code{\% java usr.vim.Vim control-config.xml}

\noindent This will start the Global Controller on the local host.
The class \texttt{usr.vim.Vim} is a subclass of the GlobalController,
with identical functionality.

It with automatically start the Local Controllers and the Routers in the
relevant places.  The links between the Routers will be created, under
the control of the Global Controller, based on the configuration file.



\subsection{Configuration for Controlled Setups}

The Controlled Setups start with only a GlobalController and
LocalControllers, and they wait for external control via the REST API
to configure the virtual topology.
The following is a configuration, which has 4 main sections:

\begin{itemize}
\item GlobalController -- with configuration options for the GlobalController itself
\item LocalController -- with configuration details for LocalControllers
\item EventEngine -- setup for an EventEngine
\item RouterOptions -- with configuration for the virtual Routers
\end{itemize}

\noindent This configuration starts the Global
Controller on port 8888 of the local host,
with a placement engine of class \texttt{usr.globalcontroller.LeastUsedLoadBalancer}.
The Lattice monitoring is configured with 5 different data consumers.
The Global Controller starts
 Local Controllers on 3 other hosts: host1, host2, and host3.
The Local Controllers will listen on
port 10000 of each host.

The configuration for the routers is held in the file
\texttt{scripts/routeroptions.xml}.




\noindent All of the configuration options can be found in section
\ref{globalcontroller:config}.


\needspace{5\baselineskip}

\begin{lstlisting}[language=config, caption=control-wait-config.xml]
<SimOptions>
  <GlobalController>
     <Port>8888</Port>
     <StartLocalControllers>true</StartLocalControllers>

     <PlacementEngineClass>usr.globalcontroller.LeastUsedLoadBalancer</PlacementEngineClass>
     <Monitoring>
       <LatticeMonitoring>true</LatticeMonitoring>
       <MonitoringPort>7799</MonitoringPort>

       <Consumer> 
         <Name>usr.globalcontroller.HostInfoReporter</Name>
       </Consumer>
       <Consumer>
         <Name>usr.globalcontroller.NetIFStatsReporter</Name>
       </Consumer>
       <Consumer>
         <Name>usr.globalcontroller.RouterAppsReporter</Name>
       </Consumer>
       <Consumer>
         <Name>usr.globalcontroller.ThreadGroupListReporter</Name>
       </Consumer>
       <Consumer>
         <Name>usr.globalcontroller.ThreadListReporter</Name>
       </Consumer>
     </Monitoring>
  </GlobalController>

  <LocalController>
     <Name>host1</Name>
     <Port>10000</Port>
     <LowPort>11001</LowPort>
     <HighPort>12000</HighPort>
     <MaxRouters>100</MaxRouters>
  </LocalController>

  <LocalController>
     <Name>host2</Name>
     <Port>10000</Port>
     <LowPort>12001</LowPort>
     <HighPort>13000</HighPort>
     <MaxRouters>100</MaxRouters>
  </LocalController>

  <LocalController>
     <Name>host3</Name>
     <Port>10000</Port>
     <LowPort>13001</LowPort>
     <HighPort>14000</HighPort>
     <MaxRouters>100</MaxRouters>
  </LocalController>

  <EventEngine>
     <Name>Empty</Name>
     <EndTime>86400</EndTime> 
  </EventEngine>

  <RouterOptions>
      scripts/routeroptions.xml
  </RouterOptions>

</SimOptions>

\end{lstlisting}


\noindent  In the above configuration for \texttt{GlobalController}
the following options have been set:

{
\small

\begin{longtable}{ | p{3.4cm} | p{3.1cm} | p{7.3cm} | }

\hline
\textbf{Option} & \textbf{Value} & \textbf{Description} \\
\hline
Port & 8888 & The GlobalController will listen for REST API commands
on port 8888. \\
\hline
StartLocalControllers & true & The GlobalController should start
LocalControllers if they are not already running. \\
\hline
PlacementEngineClass & usr.globalcontroller. LeastUsedLoadBalancer &
The name of the class for the Placement Engine. \\
\hline
Monitoring & More config options & If required, the details for
Monitoring. \\
\hline
$\rightarrow$ LatticeMonitoring & true & Should the GlobalController start the Lattice
Monitoring sub-system. \\
\hline
$\rightarrow$ MonitoringPort & 7799 & The Lattice Monitoring sub-system will listen
on port 7799 for monitoring data. \\
\hline
$\rightarrow$ Consumer & usr.globalcontroller. HostInfoReporter & A monitoring
consumer that collects data from LocalControllers. \\
\hline
$\rightarrow$ Consumer & ... & Other monitoring consumers. \\
\hline

\end{longtable}

\normalsize
}

Now we present the configuration for \texttt{LocalController}
blocks.  A GlobalController needs one or more LocalController definitions.
In the above example, there are 3 LocalController definitions, with
the following  options having been set: 

{
\small

\begin{longtable}{ | p{3.4cm} | p{3.1cm} | p{7.3cm} | }

\hline
\textbf{Option} & \textbf{Value} & \textbf{Description} \\
\hline
Name & host1 & The name of the host that the LocalController should be
started on. \\
\hline
Port & 10000 & The LocalController will listen 
on port 10000 for commands from the GlobalController. \\
\hline
LowPort & 11001 & The LocalController will start a virtual router,
with the router management port in a range, with this the low end. \\
\hline
Highort & 12000 & The LocalController will start a virtual router,
with the router management port in a range, with this the high end. \\
\hline
MaxRouters & 100 & The maximum number of virtual routers to start in
this host. \\
\hline

\end{longtable}

\normalsize
}
  

\noindent Now we present the configuration for \texttt{EventEngine}
block. The setup for externally controlled configurations
has the following  options set: 

{
\small

\begin{longtable}{ | p{3.4cm} | p{3.1cm} | p{7.3cm} | }

\hline
\textbf{Option} & \textbf{Value} & \textbf{Description} \\
\hline
Name & Empty & The EventEngine to execute. Empty means there is no
events being generated. \\
\hline
EndTime & 86400 & The amount of time the system will run -- even if
there are no events being created.  In this case it is \emph{86400
  seconds}, which is \emph{1440 minutes}, or \emph{24 hours}. \\
\hline


\end{longtable}

\normalsize
}
  
\noindent Now we present the configuration for \texttt{RouterOptions}
block. This always has the name of another configuration file which
contains the actual router options. In this case it is the file:
\texttt{scripts/routeroptions.xml}.
We will now look at the router configuration file: routeroptions.xml.

\needspace{5\baselineskip}

\begin{lstlisting}[language=config, caption=routeroptions.xml]

<RouterOptions>
    <Monitoring>
      <LatticeMonitoring>true</LatticeMonitoring>
      <Probe>
        <Name>usr.router.NetIFStatsProbe</Name>
        <Rate>1000</Rate>
      </Probe>
      <Probe>
        <Name>usr.router.ThreadListProbe</Name>
        <Rate>1000</Rate>
      </Probe>
      <Probe>
        <Name>usr.router.ThreadGroupListProbe</Name>
        <Rate>5000</Rate>
      </Probe>
      <Probe>
        <Name>usr.router.AppListProbe</Name>
        <Rate>5000</Rate>
      </Probe>
    </Monitoring>

    <APManager>
        <Name>None</Name>
    </APManager> 

</RouterOptions>


\end{lstlisting}

\noindent  In the above configuration for \texttt{RouterOptions}
the following options have been set:

{
\small

\begin{longtable}{ | p{3.4cm} | p{3.1cm} | p{7.3cm} | }

\hline
\textbf{Option} & \textbf{Value} & \textbf{Description} \\
\hline
Monitoring & More config options & If required, the details for
Monitoring. \\
\hline
$\rightarrow$ LatticeMonitoring & true & Should the Router start the Lattice
monitoring and send data from the defined probes. \\
\hline
$\rightarrow$ Probe & usr.router. NetIFStatsProbe & A monitoring
probe that collects \emph{Network Interface} packets data from a Router . \\
\hline
$\rightarrow$ Probe & ... & Other monitoring probes. \\
\hline
APManager & More config options & If required, the details for
an Aggregation Point Manager. \\
\hline
 $\rightarrow$  Name & None & This setup does not use Aggregation Points.
\\
\hline
\end{longtable}

\normalsize
}



\subsection{Configuration for Probabilistic Setups}

This setup is configured to use a single host, with the
GlobalController and the LocalController both residing on the localhost.
The following is a minimal configuration, which has 4 main sections:

\begin{itemize}
\item GlobalController which defines values for the Global Controller
\item LocalController which defines values for Local Controllers
\item EventEngine which defines values for an event engine
  that drives the creation of routers and links 
\item RouterOptions which defines values for the routers.
\end{itemize}

\noindent This configuration starts the Global
Controller on port 8888 of the local host, and that it should start
a Local Controller.
The Local Controller will be on
port 10000 of the local host.
It also tells the LocalController to allocate Routers which listen on
a range of ports, between port 11001 and 12000, and that there should
be a maximum of 50 routers on localhost.

The EventEngine section configures the Probabilistic engine to
generate new Router and Connection requests.  The engine will execute
for 600 seconds, and will get the probability distribution information
from the file \texttt{probdists.xml}.

Finally, the configuration for the routers is held in the file \texttt{routeroptions.xml}.

\needspace{5\baselineskip}

\begin{lstlisting}[language=config, caption=control-probabalistic-config.xml]
<SimOptions>

  <GlobalController>
     <Port>8888</Port>
     <StartLocalControllers>true</StartLocalControllers>
     <ConnectedNetwork>true</ConnectedNetwork>
  </GlobalController>

  <LocalController>
     <Name>localhost</Name>
     <Port>10000</Port>
     <LowPort>11001</LowPort>
     <HighPort>12000</HighPort>
     <MaxRouters>50</MaxRouters>
  </LocalController>

  <EventEngine>
     <Name>Probabilistic</Name>
     <EndTime>600</EndTime>
     <Parameters>probdists.xml</Parameters>
  </EventEngine>

  <RouterOptions>
      routeroptions.xml
  </RouterOptions>

</SimOptions>
\end{lstlisting}

\noindent The options set for the configuration of the
\texttt{GlobalController} and the \texttt{LocalController} are as
described earlier.
But we present the configuration for \texttt{EventEngine}
block, as this is different here.

%\needspace{5\baselineskip}

{
\small

\begin{longtable}{ | p{3.4cm} | p{3.1cm} | p{7.3cm} | }

\hline
\textbf{Option} & \textbf{Value} & \textbf{Description} \\
\hline
Name & Probabilistic & The EventEngine is the Probabilistic event
engine which generates events in a probabilistic manner. \\
\hline
EndTime & 600 & The amount of time the system will run -- even if
there are no events being created.  In this case it is \emph{600
  seconds}, which is \emph{10 minutes}. \\
\hline
Parameters & probdists.xml & The name of the file to get the
probability functions and coefficients for the probabilistic event engine. \\
\hline


\end{longtable}

\normalsize
}


\noindent The probabilistic event engine is defined in the class
\texttt{usr.engine.ProbabilisticEventEngine}.  It expects extra
probability distribution configuration to be defined. For this example
the following is defined:

\needspace{5\baselineskip}

\begin{lstlisting}[language=config, caption=probdists.xml]
<ProbabilisticEngine>
      <NodeBirthDist>
            <ProbElement>
                 <Type>Exponential</Type>
                 <Weight>1.0</Weight>
                 <Parameter>3.0</Parameter>
            </ProbElement>
      </NodeBirthDist>

      <NodeDeathDist>
            <ProbElement>
                 <Type>Exponential</Type>
                 <Weight>0.7</Weight>
                 <Parameter>60</Parameter>
            </ProbElement>
            <ProbElement>
                 <Type>LogNormal</Type>
                 <Weight>1.0</Weight>
                 <Parameter>7.0</Parameter>
                 <Parameter>1.5</Parameter>
            </ProbElement>
      </NodeDeathDist>

      <LinkCreateDist>
            <ProbElement>
                 <Type>PoissonPlus</Type>
                 <Weight>1.0</Weight>
                 <Parameter>1.5</Parameter>
                 <Parameter>1.0</Parameter>
            </ProbElement>
      </LinkCreateDist>

      <Parameters>
          <PreferentialAttachment>true</PreferentialAttachment>
      </Parameters>
</ProbabilisticEngine>
\end{lstlisting}


\noindent Here we present the configuration for \texttt{ProbabilisticEngine}
block.

%\needspace{5\baselineskip}

{
\small

\begin{longtable}{ | p{3.4cm} | p{3.1cm} | p{7.3cm} | }

\hline
\textbf{Option} & \textbf{Value} & \textbf{Description} \\
\hline
NodeBirthDist &  & Options for the probabilistic creation of new
virtual routers. \\
\hline
NodeDeathDist &  & Options for the probabilistic lifetime of a
virtual router. \\
\hline
LinkCreateDist &  & Options for the probabilistic creation of new
virtual links. \\
\hline
$\rightarrow$ ProbElement & & The specification for this probability
distribution.\\
\hline
$\rightarrow$ Type & Exponential & Use the Exponential distribution. \\
\hline
$\rightarrow$ Weight & 1.0 & A weight for thr distribution. \\
\hline
$\rightarrow$ Parameter & 3.0 & A coefficient parameter for thr distribution. \\
\hline
Parameters &  &  Optional extra parameters. \\
\hline
$\rightarrow$ PreferentialAttachment & true & Use a preferential
attachment strategy when adding new links. \\
\hline
\end{longtable}

\normalsize
}


\noindent The values for the $Type$ of probability distribution are:
\texttt{Exponential}, 
\texttt{LogNormal},
\texttt{PoissonPlus}, and
\texttt{Uniform}.

We will now look at the router configuration.


\needspace{5\baselineskip}

\begin{lstlisting}[language=config, caption=routeroptions-with-ap.xml]
<RouterOptions>

    <APManager>
        <Name>Pressure</Name>
        <MaxAPs>100</MaxAPs>
        <MinAPs>1</MinAPs>
        <MaxAPWeight>5</MaxAPWeight>
        <MinPropAP>0.1</MinPropAP>
    </APManager> 

</RouterOptions>
\end{lstlisting}

\noindent Here we present the configuration for \texttt{RouterOptions}
block.

%\needspace{5\baselineskip}

{
\small

\begin{longtable}{ | p{3.4cm} | p{3.1cm} | p{7.3cm} | }

\hline
\textbf{Option} & \textbf{Value} & \textbf{Description} \\
\hline
APManager & More config options & If required, the details for
an Aggregation Point Manager. \\
\hline
 $\rightarrow$  Name & Pressure & This setup uses the Pressure
Aggregation Point algorithm. \\
\hline
 $\rightarrow$  MaxAPs & 100 & The maximum number of APs in the network. \\
\hline
 $\rightarrow$  MinAPs & 1  & The minimum number of APs in the network.  \\
\hline
 $\rightarrow$  MaxAPWeight & 5 &  \\
\hline
 $\rightarrow$  MinPropAP &  0.1 &  \\
\hline


\end{longtable}

\normalsize
}

\noindent The values for the $Name$ of the Aggregation Point Manager strategy are:
\texttt{Pressure}, 
\texttt{Random},
\texttt{HotSpot}, and
\texttt{None}.

%%\subsubsection{Execution of Probabilistic Setups}

%%\textit{TODO}

\subsection{Stopping}

The Global Controller will keep running for the number of seconds
defined in $EventEngine \rightarrow EndTime$ specification.
In the Configuration for Controlled Setup example it was set to 86400
seconds, and in the Probabilistic Setup example it was set to 600
seconds.
Once the specified time has elapsed the GlobalController will go into
its shutdown phase and will
automatically stop all app, remove all virtual links, terminate all
virtual routers,  and stop all LocalControllers.

It is possible to stop the GlobalController under software control by
sending a special command via the REST API.
This call acts the same way as the time elapsing and will go into the
shutdown phase.


\begin{verbatim}
GET http://host:8888/command/SHUT_DOWN
\end{verbatim}

