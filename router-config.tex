\subsection{Router Configuration Options}

% The table of router configuration options

% This needs a few passes to layout correctly

{
  \renewcommand{\arraystretch}{1.6} 
  \renewcommand{\tabcolsep}{1.1ex}


\small

% longtable lines
\begin{longtable}{ | p{7.5cm} | p{6.5cm} | }

 \caption{Router Configuration Options} \\
 \hline \multicolumn{2}{| c |} {\textbf{Configuration Options}}\\
 \hline  {\textbf{Field}} & {\textbf{Description}} \\
 \endhead



\hline
\footnotesize{\texttt{<RouterOptions>}} & The top level node for all
configuration options to the Router. \\
\hline \hline

% Router

\hline
\footnotesize{\texttt{<Router>} (optional)}} &
The node for specifying the actual Router class.
\\

\hline
\footnotesize{\texttt{<Router> $\rightarrow$ <RouterClass>}} &
The class name of the virtual router to start.
\newline
String: The default is \texttt{usr.router. Router}.
 \\

% RoutingParameters

\hline
\footnotesize{\texttt{<RoutingParameters>}} &
The node for specific options for  the routing.
\\

\hline
\footnotesize{\texttt{<RoutingParameters> $\rightarrow$ <LinkType>}} &
What kind of transport is used for the virtual link: either UDP or TCP.
\newline
Default option is \texttt{TCP}.
 \\

\hline
\footnotesize{\texttt{<RoutingParameters> $\rightarrow$ <MaxCheckTime>}} &
How many millis to wait between checks of routing table.
\newline
Example option is \texttt{60000}.
 \\

\hline
\footnotesize{\texttt{<RoutingParameters> $\rightarrow$ <MinNetIFUpdateTime>}} &
Shortest interval between routing updates down given NetIF.
\newline
Example option is \texttt{5000}.
 \\

\hline
\footnotesize{\texttt{<RoutingParameters> $\rightarrow$ <MaxNetIFUpdateTime>}} &
Longest interval between routing updates down given NetIF.
\newline
Example option is \texttt{30000}.
\\
\hline
\footnotesize{\texttt{<RoutingParameters> $\rightarrow$ <DatagramType>}} &
\newline
String: Value is class name which will be datagram class used by network.
E.g. usr.net.GIDDatagram.  String must point to valid class implementing
Datagram interface.
\\
\hline 
\hline

% APManager

\hline 
\footnotesize{\texttt{<APManager>}} &
The node for specific options for  the APManager. \\

\hline
\footnotesize{\texttt{<APManager> $\rightarrow$ <Name>}} &
The name of the Aggregation Point selection algorithm.
\newline
Options are \texttt{None}, \texttt{Pressure}, \texttt{Random}, or \texttt{HotSpot}
 \\

\hline
\footnotesize{\texttt{<APManager> $\rightarrow$ <MaxAPs>}} &
The maximum number of APs allowed. \\

\hline
\footnotesize{\texttt{<APManager> $\rightarrow$ <MinAPs>}} &
The minimum number of APs allowed. \\

\hline
\footnotesize{\texttt{<APManager> $\rightarrow$ <RouterConsiderTime>}} &
Time router reconsiders APs. \\

\hline
\footnotesize{\texttt{<APManager> $\rightarrow$ <ControllerConsiderTime>}} &
Time controller reconsiders APs. \\

\hline
\footnotesize{\texttt{<APManager> $\rightarrow$ <MaxAPWeight>}} &
Maximum link weight an AP can be away. \\

\hline
\footnotesize{\texttt{<APManager> $\rightarrow$ <MinPropAP>}} &
Minimum proportion of APs in the network. \\

\hline
\footnotesize{\texttt{<APManager> $\rightarrow$ <MonitorType>}} &
Maximum proportion of APs in the network. \\
\hline 
\hline

% Output

\hline
\footnotesize{\texttt{<Output>}} &
The node for specific options for  generating output from a run. \\
\hline

\end{longtable}


\normalsize
}
