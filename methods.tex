The following table shows the standard Java interface for
DatagramSockets. It shows the method, what the method does, and
whether it is supported by the User Space Routing framework.


% The table of methods

% This needs a few passes to layout correctly
{
  \renewcommand{\arraystretch}{1.6} 
  \renewcommand{\tabcolsep}{1.1ex}



\small

% longtable lines
\begin{longtable}{ | p{7cm} | p{5.6cm} | p{1.2cm} | }

 \caption{DatagramSocket Method Support} \\

 \hline \multicolumn{3}{| c |} {\textbf{Method Support}}\\

 \hline  {\textbf{Method}} & {\textbf{Description}} & {\textbf{In\linebreak[4]USR}}\\

 \endhead


\hline
\footnotesize{\texttt{DatagramSocket()}} & 
          Constructs a datagram socket and binds it to any available
          port on the local host machine.& 
          YES \\

%protected	DatagramSocket(DatagramSocketImpl impl) 
%
%          Creates an unbound datagram socket with the specified DatagramSocketImpl.


\hline
\footnotesize{\texttt{DatagramSocket(int port)}} & 
          Constructs a datagram socket and binds it to the specified
          port on the local host machine. &
          YES \\

\hline
\footnotesize{\texttt{DatagramSocket(int port, InetAddress addr)}} & 
          Creates a datagram socket, bound to the specified local
          address. &
          REPL \\
\hline
\multicolumn{1}{| l } { \emph{Replacement} $\Rightarrow$ } &
\multicolumn{2}{ l |} {\footnotesize{\texttt{DatagramSocket(Address addr, int port)}}} \\

\hline
\footnotesize{\texttt{DatagramSocket(SocketAddress bindaddr)}} & 
          Creates a datagram socket, bound to the specified local
          socket address. &
          NO \\

\hline
\footnotesize{\texttt{void bind(SocketAddress addr)}} & 
          Binds this DatagramSocket to a specific address \& port. &
          REPL \\
\hline
\multicolumn{1}{| l } { \emph{Replacement} $\Rightarrow$ } &
\multicolumn{2}{ l |} {\footnotesize{\texttt{void bind(int port)}}} \\

\hline
\footnotesize{\texttt{void close()}} &
          Closes this datagram socket. &
          YES \\

\hline
\footnotesize{\texttt{void connect(InetAddress addr, int port)}} &
          Connects the socket to a remote address for this socket. &
          REPL \\

\hline
\multicolumn{1}{| l } { \emph{Replacement} $\Rightarrow$ } &
\multicolumn{2}{ l |} {\footnotesize{\texttt{void connect(Address address, int port)}}} \\

\hline 
\footnotesize{\texttt{void connect(SocketAddress addr)}} &
          Connects this socket to a remote socket address (IP address + port number).&
          YES \\

\hline
\footnotesize{\texttt{void disconnect()}} &
          Disconnects the socket. &
          YES \\
\hline
\footnotesize{\texttt{InetAddress getInetAddress()}} &
          Returns the address to which this socket is connected. &
          REPL \\
\hline
\multicolumn{1}{| l } { \emph{Replacement} $\Rightarrow$ } &
\multicolumn{2}{ l |} {\footnotesize{\texttt{Address getRemoteAddress()}}} \\

\hline
\footnotesize{\texttt{InetAddress getLocalAddress()}} &
          Gets the local address to which the socket is bound. &
          YES \\
\hline
\multicolumn{1}{| l } { \emph{Replacement} $\Rightarrow$ } &
\multicolumn{2}{ l |} {\footnotesize{\texttt{Address getLocalAddress()}}} \\

\hline
\footnotesize{\texttt{int getLocalPort()}} &
          Returns the port number on the local host to which this
          socket is bound. &
          YES \\
\hline
\footnotesize{\texttt{SocketAddress getLocalSocketAddress()}} &
          Returns the address of the endpoint this socket is bound to,
          or null if it is not bound yet. &
          NO \\
\hline
\footnotesize{\texttt{int getPort()}} &
          Returns the port for this socket. &
          YES \\
\hline
\footnotesize{\texttt{SocketAddress getRemoteSocketAddress()}} &
          Returns the address of the endpoint this socket is connected
          to, or null if it is unconnected. &
          NO \\
\hline
\footnotesize{\texttt{boolean isBound()}} &
          Returns the binding state of the socket. &
          YES \\
\hline
\footnotesize{\texttt{boolean isClosed()}} &
          Returns whether the socket is closed or not. &
          YES \\
\hline
\footnotesize{\texttt{boolean isConnected()}} &
          Returns the connection state of the socket. &
          YES \\
\hline
\footnotesize{\texttt{void receive(DatagramPacket p)}} &
          Receives a datagram packet from this socket. &
          REPL \\
\hline
\multicolumn{1}{| l } { \emph{Replacement} $\Rightarrow$ } &
\multicolumn{2}{ l |} {\footnotesize{\texttt{Datagram receive()}}} \\

\hline
\footnotesize{\texttt{void send(DatagramPacket p)}} &
          Sends a datagram packet from this socket. &
          REPL \\
\hline
\multicolumn{1}{| l } { \emph{Replacement} $\Rightarrow$ } &
\multicolumn{2}{ l |} {\footnotesize{\texttt{send(Datagram dg)}}} \\

\hline
\hline
\multicolumn{3}{| p{12cm} |} {
\newline
\textit{The DatagramSocket of USR does not support any Socket Options 
or Traffic Class functions, and so none of these methods are
supported.}
\newline
}\\

%\hline
%static void setDatagramSocketImplFactory(DatagramSocketImplFactory fac)}} &
%          Sets the datagram socket implementation factory for the
%          application. &
%          NO \\
\hline
\footnotesize{\texttt{DatagramChannel getChannel()}} &
          Returns the unique DatagramChannel object associated with
          this datagram socket, if any. &
          NO \\
\hline
\footnotesize{\texttt{boolean getBroadcast()}} &
          Tests if SO\_BROADCAST is enabled. &
          NO \\
\hline
\footnotesize{\texttt{void setBroadcast(boolean on)}} &
          Enable/disable SO\_BROADCAST. &
          NO \\
\hline
\footnotesize{\texttt{boolean getReuseAddress()}} &
          Tests if SO\_REUSEADDR is enabled. &
          NO \\
\hline
\footnotesize{\texttt{int getSendBufferSize()}} &
          Get value of the SO\_SNDBUF option for this DatagramSocket,
          that is the buffer size used by the platform for output on
          this DatagramSocket. &
          NO \\
\hline
\footnotesize{\texttt{int getSoTimeout()}} &
          Retrive setting for SO\_TIMEOUT. &
          NO \\
\hline
\footnotesize{\texttt{int getReceiveBufferSize()}} &
          Get value of the SO\_RCVBUF option for this DatagramSocket,
          that is the buffer size used by the platform for input on
          this DatagramSocket. &
          NO \\
\hline
\footnotesize{\texttt{void setReceiveBufferSize(int size)}} &
          Sets the SO\_RCVBUF option to the specified value for this
          DatagramSocket. &
          NO \\
\hline
\footnotesize{\texttt{void setReuseAddress(boolean on)}} &
          Enable/disable the SO\_REUSEADDR socket option. &
          NO \\
\hline
\footnotesize{\texttt{void setSendBufferSize(int size)}} &
          Sets the SO\_SNDBUF option to the specified value for this
          DatagramSocket. &
          NO \\
\hline
\footnotesize{\texttt{void setSoTimeout(int timeout)}} &
          Enable/disable SO\_TIMEOUT with the specified timeout, in
          milliseconds. &
          NO \\
\hline
\footnotesize{\texttt{int getTrafficClass()}} &
          Gets traffic class or type-of-service in the IP datagram
          header for packets sent from this DatagramSocket. &
          NO \\
\hline
\footnotesize{\texttt{void setTrafficClass(int tc)}} &
          Sets traffic class or type-of-service octet in the IP
          datagram header for datagrams sent from this
          DatagramSocket. & NO \\
\hline 

\end{longtable}

\normalsize

\label{usr:method-table}

}
