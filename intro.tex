This document gives an overview of the \emph{VLSP}
framework - sometimes called  \emph{User Space Routing} as it does
everything in user-space.
We have designed a  testbed called
the \emph{Very Lightweight Network \& Service Platform} based on this framework.
It uses a set of Virtual Routers and Virtual Network Connections to
create a test  environment which can easily accommodate
 (i) fast setup and teardown of a Virtual Router,
and (ii) fast setup and teardown of a Virtual Connection.
Each Virtual Router can run small programs and service elements. 

The platform consists of a number of virtual routers running
as Java virtual machines (JVMs) across a  number of physical machines.
The routers are logically independent software entities which, as with
real routers, communicate with each other via network
interfaces.  The network traffic is made up of datagrams, which are
send to and from each router. The datagrams are not real UDP
datagrams, but are our own virtual datagrams (called USR datagrams) which are
used.

The platform has three components, the major one of 
which is the ``Router'' itself.  They are complemented by a
lightweight ``Local Controller'' which is similar to a hypervisor and has the role of sending instructions to start up 
or shutdown routers on the local machine and for routers to setup or
tear-down connections with other virtual routers.  The whole testbed
is supervised by a ``Global Controller''.  
This software entity is like a combined Virtual Infrastructure Management (VIM) / Orchestrator,
%does not exist on a real running virtual router system but,
and has the role of a
co-ordinator.  That is to say, it informs 
Local Controllers when to start up and shut down routers and when to 
connect them to each other or disconnect them from each other.
%In a real system these decisions might be made by the management
%system.


Since its inception, VLSP has been used in many experimental
situations that have benefited from its flexibility, adaptability,
lightweightness, and scalability.
VLSP achieves better simplicity and non-functional characteristics
over using a hypervisor running a standard virtual machine and
standard OS (i.e. improved scalability; lower resource utilization;
quicker startup speed; reduced heaviness; eliminate the issue where
most of the router functionality is not needed; and more networking
flexibility).  VLSP is fully distributed and modular. This has
been achieved, as the management components as well as the host
controllers and the virtual entities themselves can be deployed across
any number of physical hosts interacting via REST calls.


\noindent From the experimental usage of VLSP we have determined:
\begin{description}[leftmargin=1em,labelindent=0,itemsep=3pt]
\item \textit{What is VLSP good for?}

\begin{itemize}
\item Testing and evaluating alternative SDN / NFV scenarios
\item Mid scale tests of network software written in Java (100s and likely 
1000s of virtual routers).
\item Testing software robustness to ``unexpected" network conditions
(sudden ``rude" start up/shut down exposes software deficiencies).
\item Testing software robustness to unreliable networks.
\item A compromise between simulation (realism questionable) and large
testbed (requires many physical machines).
\item Comparing simulation with testbed results
\end{itemize}

\item \textit{What is VLSP \emph{not} yet good for?}

\begin{itemize}
\item It is not optimized for forwarding performance, compared to a real router it routes packets at a slower speed and uses more overhead (i.e. due to the focus on the support of new network management and control features).
\item It is difficult to support facilities and protocols that rely on maximum bandwidth
  calculations, e.g. traffic engineering algorithms estimating
  the link bandwidth.
  \item The direct interaction with or the driving of hardware interfaces
  is out not currently addressed. % in the current implementation.
%  has space for performance improvements.
\end{itemize}

\end{description}



\subsection{Motivation}

There were many motivations for designing and building the \emph{User Space Routing}
framework.  These were accumulated from experience on various research
projects, including RESERVOIR which investigated running services in virtual machines, and
the AutoI project, which investigated the virtualization of network
elements.
It was found that the use of a hypervisor and the associated virtual
machines did work as expected and as required, however, there were
some issues that hindered various experimental situations.

We found that using a hypervisor and virtual machines added only 5\%
to 10\% overhead to operations, compared to running the same
operations in the physical machine, which is most cases was entirely
acceptable.  The small loss of efficiency was easily overcome by the
flexibility of having virtual machines.
In terms of experimental and research issues, some were general issues
and others were specific to the domain. We found the following general
issues:

\begin{itemize}[itemsep=1ex]
\item the number of virtual machines that can run on a physical host
  is limited.  This can be due to the actual resources of the physical
  machine that need to be shared (such as the number of cores and the
  amount of memory available), together with the switching
  capabilities of the hypervisor.

\item the speed of startup of a virtual machine can be quite
  slow. Although virtual machines boot up in the same order of
  magnitude as a physical host, there are extra layers and
  inefficiencies that slow them down.  Also, if many virtual
  machines are started concurrently, then we observe that the physical
  machine and the hypervisor thrash trying to resolve resource
  utilization. 

\item the size of a virtual machine image is quite large.  A virtual
  machine has to have a disc image which contains a full operating
  system and the applications needed for the relevant tasks.  To start
  a virtual machine, the operating system needs to be booted and then
  the applications started.  So every virtualized application needs
  the overhead of a full OS.

\end{itemize}

\noindent and we found the following issues that were more domain
specific:

\begin{itemize}[itemsep=1ex]

\item in terms of virtual networks, and virtualized routers in
  particular, we observered that 95\% of the router functionality we
  never utilized in any of the experiments that were run.  Although
  software routers such a XORP and Quagga allow anyone to play and
  evaluated soft networks, the overhead of a virtual machine, with a
  full OS, and an application where only 2\% is used, seems to be an
  ineffective approach for many situations.

\item when trying to configure the IP networking of virtual machines
  and virtual routers, there are some serious hurdles.  The virtual
  machines do not talk directly to the network, but go via the
  hypervisor.  The hypervisor has various schemes for connecting
  virtual machines to the underlying network, each of which has
  different behaviour.  In most situations where experimentation of
  virtual routers is required, there needs to be a large range of IP
  addresses available.  However, this is often hard to come by.
  We found that the limits of addressing, the IP networking
  configuration, and virtual machine to virtual machine
  interoperability a hindrance to network topology and network flexibility.


\end{itemize}

\noindent It was felt that to make more progress in the area of
dynamic and virtual networking experimentation and research, we needed
to design and build a testbed that did not have these limits, but still retain 
virtual machine technology.

The main goals of the testbed over using a hypervisor running a
standard virtual machine and standard OS are to have:

\begin{itemize}
\item better scalability
\item lower resource utilization
\item quicker startup speed 
\item reduced heaviness 
\item eliminate the issue where 95\% of the router functionality not needed
\item more networking flexibility
\end{itemize}

\noindent The choice was made to write our own simple router with
simple service capabilities, in Java, that
could run in a Java Virtual Machine (the JVM).

\subsection{Benefits}

The benefits of a lightweight VM that includes a simple router and the
basic capabilities of a service component are: 

\begin{itemize}
\item it is possible to run many more routers on a host
\item it is easier to test scalability and stability
\item it is possible do enhanced monitoring and management evaluations
\item it provides a different way to do virtual networks: we can
  create arbitrary topologies using virtual routers 
\item it is possible to do more evaluations of network management and
orchestration functions by not
using complete routers and full services
\end{itemize}

\noindent In general, it is a more effective platform for
experimenting with many aspects of virtual networks, management of
virtual elements, and orchestration experiments.

The VLSP integrates and unifies management of networks,
compute, and services into a
single platform.
The whole system, including a lightweight virtual router, was
designed and built from scratch to enable flexible experimentation of
virtual infrastructures.  VLSP has the following
advantages:

\begin{description}[leftmargin=1.5em,labelindent=0,itemsep=3pt]

\item \textit{Lightweight virtual router implementation:}
to allow evaluation of diverse scenarios, including the
testing of various management lifecycle schemes.
It allows the testing of software 
robustness to ``unexpected" network/node conditions
(such as sudden start up/shut down events exposing software deficiencies),
where a virtual element can disappear at run-time under management control.
VLSP is a
positioned between simulation (realism questionable) and large testbed
(requires many servers).
Lightweight routers being deployed and managed in a distributed environment, suitable for reliability and scalability tests.

\item \textit{Distributed infrastructure implementing logically-centralized management and control:}
By employing the SDN paradigm, while
avoiding overflowing centralized components,
to support optimizations,
based on centralized decisions using a global system picture 
 derived from an integrated monitoring infrastructure based on
Lattice \cite{lattice}.
It also supports flexible and adaptable service provisioning, plus aspects
beyond traffic engineering, 
(e.g. adaptable service deployment /
operation, optimized distributed node placement etc).

%Hierarchical and distributed control components that perform scalable and logically-centralized network control, without overloading centralized software nodes.

\item \textit{Experimentation on management:} 
 to focus on the experimentation of distributed management and
control components of software-defined infrastructures rather
than data plane performance.
Along these lines,
 VLSP brings the following benefits:
 (i) more routers can be deployed
 on each host, making it easier to test scalability and stability;
 (ii) enhanced distributed management, control and monitoring evaluations can be carried out, which is difficult to achieve in a running environment using a number of deployed data centers;
(iii) arbitrary topologies can be created using virtual routers, providing a more general way to form virtual networks; 
(iii) suitable for both experimentation with virtual networks and
lightweight virtual servers; and (iv)  supports experiments with
dynamic topologies (such as migratable virtual routers)
 and mobile extensions of the network infrastructure). 

\item \textit{Support of integrated SDN and NFV environments:}
to uncover the potential of
virtualized SDN solutions and NFV deployments, in a naturally
integrated way, with common management and control
facilities.

\item \textit{JVM based Runtime Environment:} The evaluation of
 network topologies allowing hundreds or thousands of virtual
 nodes (routing or compute), using
 software written in Java. Each node resides in its own virtual
 machine, and so is independent of and secure from all other virtual nodes.
 The software execution environment, available on all nodes, allows the deployment of diverse network control components. Existing Java code can be quickly ported to run on our
 software router (our experience is from two or three hours, to a few days for relatively
 complicated software packages). Deployment of virtualized devices over heterogeneous hardware and software environments due to the very wide support of the JVM technology. 


\end{description}
