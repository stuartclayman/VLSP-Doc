In this Appendix there is a description of the REST API Component
Interactions, with explaining each request and response.  It
describes the interface and lists the resources that are exposed by
the Global Controller / Virtual Infrastructure Manager. It
is followed by description of all of the REST calls for each of the
resources that VLSP can interact with.

The Global Controller is the element that has access from
other components. The Local Controller is
not publically accesssible, and is there to support the Virtual
Infrastructure Manager in its role of managing the lifecycle of
Virtual Routers and Virtual Compute nodes. As such, we will only
document the interface of the Virtual Infrastructure Manager, as it is
the only element that can be addressed directly. 

The resources that are addressible via the Virtual Infrastructure
Manager are:

\begin{itemize}
\item Routers
\item Links
\item Applications
\end{itemize}

\noindent from any of the API calls.

\raggedbottom

\subsection*{API Calls}

The interface is grouped into API calls for:

\begin{itemize}
\item Routers
\item Links
\item Links on a Router
\item Applications on a Router
\item Router Link Stats
\item Shutdown for Routers
\end{itemize}

\noindent with information on the HTTP \emph{Call}, the arguments, the elements in the JSON \emph{Response}, and an \emph{Example} of a response.

\subsection{Router}

This section has the REST calls for Router resources

\hr
\subsubsection{Create Router}
This call creates a router with the specified args

\subsubsection*{Call}
\begin{verbatim}
POST http://host:8888/router/?args
\end{verbatim}

\subsubsection*{Args}
\begin{verbatim}
	[name] 
	[address]
	[parameters]
\end{verbatim}

\subsubsection*{Response}
\begin{verbatim}
	routerID 
	name 
	address
	mgmtPort
	r2rPort
	time
\end{verbatim}

\subsubsection*{Example}
\begin{lstlisting}[language=json]
{"routerID":1, "address":"1", "name":"Router-1", "mgmtPort":11001, "r2rPort":11002, "time":1362054061000}
\end{lstlisting}


\hr 
\subsubsection{Delete Router}
This deletes a specified router

\subsubsection*{Call}
\begin{verbatim}
	DELETE http://host:8888/router/2
\end{verbatim}

\subsubsection*{Args}
\begin{verbatim}
	Router id as final path element.
	No args defined yet.
\end{verbatim}

\subsubsection*{Response}
\begin{verbatim}
	routerID
	status
\end{verbatim}

\subsubsection*{Example}
\begin{lstlisting}[language=json]
{"routerID": 10, "status": "deleted"}
\end{lstlisting}


\hr
\subsubsection{List All Routers}
This will list all routers and return all router ids.
\subsubsection*{Call}
\begin{verbatim}
	GET http://host:8888/router/
\end{verbatim}

\subsubsection*{Args}
\begin{verbatim}
	[detail] = one of: id | all
	
	[name]
	[address]
\end{verbatim}

\subsubsection*{Response}
\begin{verbatim}
	type
	list
\end{verbatim}

\subsubsection*{Example}
\begin{lstlisting}[language=json]
{ "type": "router", "list": [1, 2, 4, 7]}
\end{lstlisting}


\hr
\subsubsection{Get Router Info}
This will get info on a specified router
\subsubsection*{Call}
\begin{verbatim}
	GET http://host:8888/router/2
\end{verbatim}

\subsubsection*{Args}
\begin{verbatim}
	Router id as final path element.
	No args defined yet.
\end{verbatim}

\subsubsection*{Response}
\begin{verbatim}
	routerID 
	name
	address
	links
	mgmtPort
	r2rPort
	time
\end{verbatim}

\subsubsection*{Example}
\begin{lstlisting}[language=json]
{"routerID":2, "address":"2", "name":"Router-2", "links":[1], "mgmtPort":11003, "r2rPort":11004,  "time":1361817254727}
\end{lstlisting}


\hr
\subsubsection{Get The Router Count}
This will get the current number of routers in the system
\subsubsection*{Call}
\begin{verbatim}
	GET http://host:8888/router/count
\end{verbatim}

\subsubsection*{Args}
\begin{verbatim}
	count is final path element.
	No args defined yet.
\end{verbatim}

\subsubsection*{Response}
\begin{verbatim}
	value
\end{verbatim}

\subsubsection*{Example}
\begin{lstlisting}[language=json]
{ "value": 5 }
\end{lstlisting}


\hr
\subsubsection{Get Maximum ID of a Router}
This will get the maximum ID of a router in the system

\subsubsection*{Call}
\begin{verbatim}
	GET http://host:8888/router/maxid
\end{verbatim}

\subsubsection*{Args}
\begin{verbatim}
	maxid is final path element.
	No args defined yet.
\end{verbatim}

\subsubsection*{Response}
\begin{verbatim}
	value
\end{verbatim}

\subsubsection*{Example}
\begin{lstlisting}[language=json]
{ "value": 42 }
\end{lstlisting}


\subsection{Link}
        
This section has the REST calls for Link resources.  It deals with all links in a network managed by the NEM.

\hr
\subsubsection{Create Link}
This will create a link between specified Routers. The link will be added to the network.
\subsubsection*{Call}
\begin{verbatim}
	POST http://host:8888/link/?args
\end{verbatim}

\subsubsection*{Args}
\begin{verbatim}
	router1
	router2
	[weight] 
	[linkName]
\end{verbatim}

\subsubsection*{Response}
\begin{verbatim}
	linkID
	linkName
	weight
	nodes
	time
\end{verbatim}

\subsubsection*{Example}
\begin{lstlisting}[language=json]
{"linkID":196612, "linkName":"Router-1.Connection-0", "nodes":[1,2], "time":1362079103160,"weight":10}
\end{lstlisting}


\hr
\subsubsection{Delete Link}
This will remove a link from the network
\subsubsection*{Call}
\begin{verbatim}
	DELETE http://host:8888/link/7
\end{verbatim}

\subsubsection*{Args}
\begin{verbatim}
	Link id as final path element.
	No args defined yet.
\end{verbatim}

\subsubsection*{Response}
\begin{verbatim}
	linkID
	status
\end{verbatim}

\subsubsection*{Example}
\begin{lstlisting}[language=json]
{"linkID": 196612, "status": "deleted"}
\end{lstlisting}

\hr
\subsubsection{List All Links}
This will list all links and return all link ids.
\subsubsection*{Call}
\begin{verbatim}
	GET http://host:8888/link/
\end{verbatim}

\subsubsection*{Args}
\begin{verbatim}
	[detail] - one of: id | all
\end{verbatim}

\subsubsection*{Response}
\begin{verbatim}
	type
	list
	[detail]
\end{verbatim}

\subsubsection*{Example}
\begin{lstlisting}[language=json]
{"type": "link", "list": [1, 2, 5]} 
	
	
Where 'detail=all'
	
{"type":"link", "list":[2293796,196612,1835044,1179673,655376,262153],
  "detail":[{"id":2293796,"name":"Router-5.Connection-0","nodes":[5,6], 
		"time":1361989254649,"weight":10},
		{"id":196612,"name":"Router-1.Connection-0","nodes":[1,2],
		"time":1361989254649,"weight":10},
		{"id":1835044,"name":"Router-4.Connection-0","nodes":[4,6],
		"time":1361989254649,"weight":10},
		{"id":1179673,"name":"Router-3.Connection-0","nodes":[3,5],
		"time":1361989254649,"weight":10},
		{"id":655376,"name":"Router-2.Connection-0","nodes":[2,4],
		"time":1361989254649,"weight":10},
		{"id":262153,"name":"Router-1.Connection-1","nodes":[1,3],
		"time":1361989254649,"weight":10}}],
}
\end{lstlisting}


\hr
\subsubsection{Get Link Info}
This will get info on a specified link
\subsubsection*{Call}
\begin{verbatim}
	GET http://host:8888/link/8
\end{verbatim}

\subsubsection*{Args}
\begin{verbatim}
	Link id as final path element.
	No args defined yet.
\end{verbatim}

\subsubsection*{Response}
\begin{verbatim}
	linkID
	linkName
	weight
	nodes
	time
\end{verbatim}

\subsubsection*{Example}
\begin{lstlisting}[language=json]
{"linkID":8, "linkName":"Router-5.Connection-0", "weight":10, "nodes":[5,6], "time":1362079709109}
\end{lstlisting}


\hr
\subsubsection{Set Link Weight}
This will set a new weight on a specified link
\subsubsection*{Call}
\begin{verbatim}
	PUT http://host:8888/link/8?weight=30
\end{verbatim}

\subsubsection*{Args}
\begin{verbatim}
	Link id as final path element.
	weight
\end{verbatim}

\subsubsection*{Response}
\begin{verbatim}
	linkID
	linkName
	weight
	nodes
	time
\end{verbatim}

\subsubsection*{Example}
\begin{lstlisting}[language=json]
{"linkID":8, "linkName":"Router-5.Connection-0", "weight":30, "nodes":[5,6], "time":1362079709109}
\end{lstlisting}


\hr
\subsubsection{Get The Link Count}
This will get the current number of links in the system
\subsubsection*{Call}
\begin{verbatim}
	GET http://host:8888/link/count
\end{verbatim}

\subsubsection*{Args}
\begin{verbatim}
	count is final path element.
	No args defined yet.
\end{verbatim}

\subsubsection*{Response}
\begin{verbatim}
	value
\end{verbatim}

\subsubsection*{Example}
\begin{lstlisting}[language=json]
{ "value": 17 }
\end{lstlisting}

\subsection{Links on a Router}

This section has the REST calls for Link resources attached to a specified Router. It only deals with Links for the specified Router.

\hr
\subsubsection{List Router Links}
This call lists all links on a specified router
\subsubsection*{Call}
\begin{verbatim}
	GET http://host:8888/router/9/link/
\end{verbatim}

\subsubsection*{Args}
\begin{verbatim}
	[attr]  - one of:  id | name | weight | connected
\end{verbatim}

\subsubsection*{Response}
\begin{verbatim}
	routerID
	type
	list
\end{verbatim}

\subsubsection*{Example}
\begin{lstlisting}[language=json]
{"routerID": 9,"type":" link", "list":[9830481,7078009,6488164]}
\end{lstlisting}



\hr
\subsubsection{Get Router Link Info}
This will get info on a specified link on a specific router
\subsubsection*{Call}
\begin{verbatim}
	GET http://host:8888/router/9/link/12
\end{verbatim}

\subsubsection*{Args}
\begin{verbatim}
	Link id as final path element.
	No args defined yet.
\end{verbatim}

\subsubsection*{Response}
\begin{verbatim}
	linkID
	linkName
	weight
	nodes
	time
\end{verbatim}

\subsubsection*{Example}
\begin{lstlisting}[language=json]
{"linkID":12, "linkName":"Router-9.Connection-0" ,"weight":10, "nodes":[9,10], "time":1362080281512}
\end{lstlisting}


\subsection{Applications on a Router}

This section has the REST calls for App resources

\hr
\subsubsection{Create App}
This call creates an app on a specified router
\subsubsection*{Call}
\begin{verbatim}
	POST http://host:8888/router/7/app/?args
\end{verbatim}

\subsubsection*{Args}
\begin{verbatim}
	Router id and 'app'. 
	className
	args
\end{verbatim}

\subsubsection*{Response}
\begin{verbatim}
	appID
	appName
	classname
	args
	routerID
	starttime
	runtime
\end{verbatim}

\subsubsection*{Example}
\begin{lstlisting}[language=json]
{"appID":2, "appName":"/Router-7/App/usr.applications.Send/1", "classname":"usr.applications.Send", "args":"[8, 4000, 2500000, -s, 1024]", "routerID":7, "starttime":1362090555686,"runtime":0}
\end{lstlisting}


\hr
\subsubsection{List Apps}
This call lists all apps on a specified router
\subsubsection*{Call}
\begin{verbatim}
	GET http://host:8888/router/7/app/
\end{verbatim}

\subsubsection*{Args}
\begin{verbatim}
	Router id and 'app'. 
	None.
\end{verbatim}

\subsubsection*{Response}
\begin{verbatim}
	type
	app
\end{verbatim}

\subsubsection*{Example}
\begin{lstlisting}[language=json]
{"type": "app", "list": [2, 3]}
\end{lstlisting}



\hr
\subsubsection{Get App Info}
This will get info on a specified app on a specific router
\subsubsection*{Call}
\begin{verbatim}
	GET http://host:8888/router/7/app/2
\end{verbatim}

\subsubsection*{Args}
\begin{verbatim}
	Router id and 'app'. 
	App id as final path element.
\end{verbatim}

\subsubsection*{Response}
\begin{verbatim}
	appID
	appName
	classname
	args
	routerID
	starttime
	runtime
\end{verbatim}

\subsubsection*{Example}
\begin{lstlisting}[language=json]
{"appID":2, "appName":"/Router-7/App/usr.applications.Send/1", "classname":"usr.applications.Send", "args":"[8, 4000, 2500000, -s, 1024]", "routerID":7, "starttime":1362090555686, "runtime": 42854}
\end{lstlisting}


\subsection{Router Link Stats}

This section has the REST calls for getting the link stats for Link resources attached to a specified Router. It only deals with Links for the specified Router.

\hr
\subsubsection{Get Router Link Stats All}
This call returns link stats for links on a specified router
\subsubsection*{Call}
\begin{verbatim}
	GET http://host:8888/router/9/link_stats/
\end{verbatim}

\subsubsection*{Args}
\begin{verbatim}
	Router id and 'link_stats'. 
	None.
\end{verbatim}

\subsubsection*{Response}
\begin{verbatim}
	value
\end{verbatim}

\subsubsection*{Example}
\begin{lstlisting}[language=json]
  {"type": "link\_stats", "links": [12 30 104], "link_stats":[
      ["Router-12 Router-7.Connection-2" 358050 1714 0 21 337824 1656 228431 1078 0 0 205428 1007 0 1 0 1],
      ["Router-30 Router-7.Connection-7" 155976 715 0 5 136884 671 73858 302 0 0 53040 260 0 1 0 2],
      ["Router-104 Router-7.Connection-19" 50203 232 0 0 45900 225 2067 3 0 0 0 0 0 1 0 1] ],
    "routerID": 7}
\end{lstlisting}


\hr
\subsubsection{Get Router Link Stats}
This call returns link stats for links on a specified router to a specified router
\subsubsection*{Call}
\begin{verbatim}
	GET http://host:8888/router/9/link_stats/12
\end{verbatim}

\subsubsection*{Args}
\begin{verbatim}
	Router id and 'link_stats' and dst router id
	None.
\end{verbatim}

\subsubsection*{Response}
\begin{verbatim}
	value
\end{verbatim}

\subsubsection*{Example}
\begin{lstlisting}[language=json]
  {"type": "link\_stats", "links": [12], "link_stats": [
      ["Router-12 Router-7.Connection-2" 358050 1714 0 21 337824 1656 228431 1078 0 0 205428 1007 0 1 0 1]],
    "routerID": 7}
\end{lstlisting}


\subsection{Shutdown for Routers}

This section has the REST calls for getting info about Routers that have shut down.

\hr
\subsubsection{List All Shutdown Routers}
This will list all the routers that have been shutdown and removed from the system
\subsubsection*{Call}
\begin{verbatim}
	GET http://host:8888/removed/
\end{verbatim}

\subsubsection*{Args}
\begin{verbatim}
	None.
\end{verbatim}

\subsubsection*{Response}
\begin{verbatim}
	type
	list
\end{verbatim}

\subsubsection*{Example}
\begin{lstlisting}[language=json]
{"type":"shutdown", "list": [{"routerID":1,"time":1362133396696}, {"routerID":2,"time":1362133396696}, {"routerID":3,"time":1362133396696}]}
\end{lstlisting}

\subsection{Local Controllers}

This section has the REST calls for getting info about Local Controllers.


\hr
\subsubsection{List Local Controllers}
This will list all LocalControllers and return all LocalController ids. A LocalController id is not an interger, but a hostname:port pair.
\subsubsection*{Call}
\begin{verbatim}
	GET http://host:8888/localcontroller/
\end{verbatim}

\subsubsection*{Args}
\begin{verbatim}
Optional: [detail] = all
\end{verbatim}

\subsubsection*{Response}
\begin{verbatim}
	type
	list
\end{verbatim}

\subsubsection*{Example}
\begin{lstlisting}[language=json]
{"list":["clayone:10000"],"type":"localcontroller"}
\end{lstlisting}

\hr
\subsubsection{Get LocalController Info}
This will get info on a specific LocalController.
\subsubsection*{Call}
\begin{verbatim}
	GET http://host:8888/localcontroller/localhost:10000
\end{verbatim}

\subsubsection*{Args}
\begin{verbatim}
LocalController id as final path element.
\end{verbatim}

\subsubsection*{Response}
\begin{verbatim}
detail
\end{verbatim}

\subsubsection*{Example}
\begin{lstlisting}[language=json]
{"detail":[{"IP":"localhost/127.0.0.1", "maxRouters":100, "name":"localhost:10000", "noRouters":0, "port":10000, "routers":[], "status":"ONLINE", "usage":0}], "list":["localhost:10000"], "type":"localcontroller"}

\end{lstlisting}



\subsection{Summary Table}

{
  \renewcommand{\arraystretch}{1.6} 
  \renewcommand{\tabcolsep}{1.1ex}


\small

\newcolumntype{L}[1]{>{\raggedright\arraybackslash\hspace{0pt}}p{#1}}%This is a wrapper to make everything a certain width -left aligned columns with stuff at the top.
%\newcolumntype{L}{1}{>{\raggedright\arraybackslash}p{#1}}

\begin{longtable}{ | L{1.4cm} | >{\footnotesize}L{1.4cm} | p{6.5cm} | L{1.8cm} | L{2.1cm} | }
\hline
\textbf{Task} & {\small \textbf{HTTP Method}} & \textbf{URI} & \textbf{Args} & \textbf{Response} \\
\hline \endhead 
Create Router & POST  & \texttt{http://host:8888/router/?args} & [name] \newline
[address] \newline
[parameters]& routerID
name
address
mgmtPort
r2rPort
time \\
\hline
Delete Router & DELETE & \texttt{http://host:8888/router/2} & router ID & routerID
status \\
\hline
List All Routers & GET & \texttt{http://host:8888/router/} & [detail] = id $|$ all & type: "router"
list: [1, 2, 4, 7] \\
\hline
Get Router Info & GET & \texttt{http://host:8888/router/2} & router ID & routerID
name
address
links
mgmtPort
r2rPort
time \\
\hline
Get No Of Routers & GET & \texttt{http://host:8888/router/count} &  & value \\
\hline
Get Max Router ID & GET & \texttt{http://host:8888/router/maxid} &  & value \\
\hline
Create Link & POST & \texttt{http://host:8888/link/?args} & router1
router2
[weight]
[linkName] & linkID
linkName
weight
nodes
time \\
\hline
Delete Link & DELETE & \texttt{http://host:8888/link/7} &  & linkID 
status \\
\hline
List All Links & GET & \texttt{http://host:8888/link/} & [detail] = id $|$ all & type: "link"
list: [8, 12, 14, 27]\\
\hline
Get Link Info & GET & \texttt{http://host:8888/link/8} & link ID & linkID
linkName
weight
nodes
time \\
\hline
Set Link Weight & PUT & \texttt{http://host:8888/link/8?args} & weight & inkID
linkName
weight
nodes
time\\
\hline
Get No Of Links & GET & \texttt{http://host:8888/link/count} &  & value \\
\hline
List Router Links & GET & \texttt{http://host:8888/router/9/link/} & [attr] = id $|$ name $|$ weight $|$ connected
 & routerID: 9
type: "link"
list: [12, 14] \\
\hline
Get Router Link Info & GET & \texttt{http://host:8888/router/9/link/12} & router ID & linkID
linkName
weight
nodes
time \\
\hline
Create App & POST & \texttt{http://host:8888/router/9/app/?args} & routerID
className
args & appID
appName
classname
args
routerID
starttime
runtime \\
\hline
List Apps & GET & \texttt{http://host:8888/router/9/app/} &  & type: "app"
list: [2, 3] \\
\hline
Get App Info & GET & \texttt{http://host:8888/router/9/app/2} & app ID & appID
appName
classname
args
routerID
starttime
runtime \\
\hline
Get Router Link Stats & GET & \texttt{http://host:8888/router/9/link\_stats} &  & routerID: 9
type: "link\_stats"
links:
link\_stats:\\
\hline
Get Router Link Stats & GET & \texttt{http://host:8888/router/9/link\_stats/12} &
\newline
router ID & routerID: 9
type: "link\_stats"
links:
link\_stats:\\
\hline
List All Shutdown Routers & GET & \texttt{http://host:8888/removed/} &  & type: "shutdown"
list: [] \\
\hline
List Local Controllers & GET & \texttt{http://host:8888/localcontroller/} & [detail] = all & type: "localcontroller" list: ["clayone:10000"]
\hline
Get Local Controller Info & GET & \texttt{http://host:8888/localcontroller/clayone:10000} &
\newline
local controller ID & detail
\hline
\end{longtable}

\normalsize

}
