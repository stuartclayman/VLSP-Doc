The following table shows the Java packages
in the User Space Routing framework.

% The table of packages

% This needs a few passes to layout correctly

{
  \renewcommand{\arraystretch}{1.6} 
  \renewcommand{\tabcolsep}{1.1ex}



\small

% longtable lines
\begin{longtable}[l]{ | p{5.5cm} | p{8.5cm} | }

 \caption{Java Packages} \\
 \hline \multicolumn{2}{| c |} {\textbf{Java Packages}}\\
 \hline  {\textbf{Package}} & {\textbf{Description}} \\
 \endhead



\hline
\footnotesize{\texttt{usr.APcontroller}} & This package has classes to control allocation of aggregation points. \\
\hline
\footnotesize{\texttt{usr.applications}} & A package for user applications \\
\hline
\footnotesize{\texttt{usr.common}} & This package has utility classes. \\
\hline
\footnotesize{\texttt{usr.console}} & This package provides classes that deal with processing network connections and requests to a ManagementConsole of a component of the system. \\
\hline
\footnotesize{\texttt{usr.dcap}} & This package provides classes that
do Datagram capture directly from the network interface. \\
\hline
\footnotesize{\texttt{usr.engine}} & This package provides classes that add events into the system. \\
\hline
\footnotesize{\texttt{usr.engine.linkpicker}} & This package provides classes that chooses links based upon finding the node with some lifetime. \\
\hline
\footnotesize{\texttt{usr.events}} & This package  has the generic functions for the Event  the system. \\
\hline
\footnotesize{\texttt{usr.events.globalcontroller}} & This package
has the functions for the Event  the system within the GlobalController. \\
\hline
\footnotesize{\texttt{usr.events.vim}} & This package  has the generic interfaces for the Event  the system. \\
\hline
\footnotesize{\texttt{usr.events.vimfunctions}} & This package
has the functions for the Event subsystem run externally. \\
\hline
\footnotesize{\texttt{usr.globalcontroller}} & This package provides classes that are part of a GlobalController. \\
\hline
\footnotesize{\texttt{usr.globalcontroller.command}} & This package provides classes that implement the commands of a GlobalController. \\
\hline
\footnotesize{\texttt{usr.globalcontroller.visualization}} & This package provides classes that  will generate a visualization of the current network topology. \\
\hline
\footnotesize{\texttt{usr.interactor}} & This package provides classes that act as a client to a ManagementConsole of a component of the system. \\
\hline
\footnotesize{\texttt{usr.localcontroller}} & This package provides classes that are part of a LocalController. \\
\hline
\footnotesize{\texttt{usr.localcontroller.command}} & This package provides classes that implement the commands of a LocalController. \\
\hline
\footnotesize{\texttt{usr.logging}} & This package provides classes that are used for logging. \\
\hline
\footnotesize{\texttt{usr.model.abstractnetwork}} & This package
provides classes that present an abstract network representation. \\
\hline
\footnotesize{\texttt{usr.model.lifeEstimate}} & This package
provides classes that  produces estimates of life spans given information about node births
   and deaths. \\
\hline
\footnotesize{\texttt{usr.net}} & This package provides classes that implement networking. \\
\hline
\footnotesize{\texttt{usr.output}} & output   \\
\hline
\footnotesize{\texttt{usr.protocol}} & This package provides classes that define the protocols that are used between the components. \\
\hline
\footnotesize{\texttt{usr.router}} & This package provides classes that are part of a Router. \\
\hline
\footnotesize{\texttt{usr.router.command}} & This package provides classes that implement
the commands of a Router. \\
\hline
\footnotesize{\texttt{usr.vim}} & This package provides classes that
are part of the VIM. \\
\hline
\end{longtable}


\normalsize

}
